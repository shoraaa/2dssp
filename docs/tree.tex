\documentclass{article}
\usepackage{amsmath,amssymb}
\begin{document}
% 2D Shortest Superstring / Smallest Canvas — ILP formulation
% 2D Shortest Super-Pattern (tile containment) -- spanning-tree model
\section*{Model: 2D Shortest-SuperTile via relative-placement spanning tree}

\subsection*{Index sets and parameters}
\begin{align*}
V &= \{1,\dots,T\}  &\text{(set of tiles)}\\
\mathcal{I} &= \{1,\dots,n\} &\text{(rows inside a tile)}\\
\mathcal{J} &= \{1,\dots,n\} &\text{(cols inside a tile)}\\
\mathcal{R} &= \{(i,j,k)\ :\ 1\le i,j\le n+1,\ 1\le k\le 4\} &\text{(allowed relative placements)}\\
t_{v,a,b} &\in \{0,1\}\quad (\text{or general symbol}) &\forall v\in V,\ a\in\mathcal{I}, b\in\mathcal{J}\\
\Delta_x(r),\ \Delta_y(r) &\in \mathbb{Z}\quad \text{(offsets implied by relative placement }r\in\mathcal{R}\text{)}.
\end{align*}
\noindent\textbf{Remark:} The mapping $r=(i,j,k)\mapsto (\Delta_x,\Delta_y)$ is defined by your corner/offset convention (e.g. corner \(k\) of tile \(v\) placed into grid relative slot \((i,j)\) of tile \(u\)). Use whichever integer offsets you defined in your implementation.

\subsection*{Decision variables}
\begin{align*}
m &\in \mathbb{Z}_{+} &\text{(side length of the square canvas)}\\
x_v,y_v &\in \mathbb{Z} &\forall v\in V \quad\text{(absolute integer coordinates of the top-left corner of tile $v$ on the canvas)}\\
e_{uvr} &\in\{0,1\} &\forall u,v\in V,\ u\ne v,\ r\in\mathcal{R}
\quad\text{(1 if the tree selects a relative placement $r$ that places $v$ w.r.t.\ $u$)}\\
\pi_{uv} &\in\{0,1\} &\forall u\ne v \quad\text{(undirected tree edge indicator; can be derived from $e_{uvr}$)}.
\end{align*}

\subsection*{Objective}
Minimize the canvas size:
\[
\min\ m
\]

\subsection*{Feasibility / domain constraints}
\[
0 \le x_v \le m-n,\qquad 0 \le y_v \le m-n \qquad \forall v\in V
\]
(tiles must lie wholly inside the \(m\times m\) canvas).

\subsection*{Tree / connectivity constraints}
We enforce that the chosen relative placements form a spanning tree on \(V\).
\begin{align}
\sum_{u\in V\setminus\{v\}}\ \sum_{r\in\mathcal{R}} e_{uvr} &= 1 &\forall v\in V\setminus\{\text{root}\} 
\tag{each non-root has exactly one parent (arborescence)}\\
\sum_{v\in V}\sum_{u\in V\setminus\{v\}}\sum_{r\in\mathcal{R}} e_{uvr} &= T-1
\tag{total edges = $T-1$}\label{eq:tree_edges}
\end{align}
(Choose an arbitrary root to break symmetry; above enforces an oriented spanning tree.)

\subsection*{Position propagation (consistency) along chosen edges}
If \(e_{uvr}=1\) then the absolute positions must satisfy the relative offset implied by \(r\):
\[
e_{uvr}=1 \quad\Longrightarrow\quad
\begin{cases}
x_v = x_u + \Delta_x(r),\\[4pt]
y_v = y_u + \Delta_y(r).
\end{cases}
\tag{position propagation}
\]
(These are logical implications; linearization suggestions are below.)

\subsection*{Global pairwise compatibility constraints}
For every pair of distinct tiles \(p,q\in V\) we must ensure that whenever they overlap on the canvas, the overlapped cells are compatible (i.e., have no conflicting symbols). Let
\[
\mathcal{O}_{(a,b),(c,d)}(p,q)\ :=\ \text{the event } \big(x_p + (a-1) = x_q + (c-1)\ \wedge\ y_p + (b-1) = y_q + (d-1)\big).
\]
Then, for every \(p\ne q\) and every pair of internal cell indices \((a,b)\in\mathcal{I}\times\mathcal{J}\), \((c,d)\in\mathcal{I}\times\mathcal{J}\) we require:
\[
\text{if } \mathcal{O}_{(a,b),(c,d)}(p,q)\ \text{holds, then } t_{p,a,b} = t_{q,c,d}.
\tag{compatibility}
\]
Equivalently (compact): for all \(p\ne q\),
\[
\bigl(x_p-x_q,\ y_p-y_q\bigr) = (a-c,\ b-d)\ \Longrightarrow\ t_{p,a,b}=t_{q,c,d}.
\]

\subsection*{Alternative formulation using precomputed pairwise compatibility}
If you precompute whether two tiles \(p,q\) are pairwise compatible under a displacement \((\delta_x,\delta_y)\), define the binary parameter
\[
\mathrm{Comp}_{pq}^{\delta_x,\delta_y}=\begin{cases}
1 &\text{if } \forall (a,b),(c,d)\ \text{with } a+\delta_x=c,\ b+\delta_y=d \text{ we have } t_{p,a,b}=t_{q,c,d},\\
0 &\text{otherwise.}
\end{cases}
\]
Then impose for all \(p\ne q\):
\[
\mathrm{Comp}_{pq}^{x_p-x_q,\ y_p-y_q} = 1.
\]
(Again this is a logical constraint; it says the relative offset produced by \((x_p,y_p)\) must be one of the compatible offsets.)

\subsection*{Remarks on implementability and linearization}
\begin{itemize}
  \item The model above uses logical implications (if \(e_{uvr}=1\) then equalities on \(x,y\); and equality tests like \(x_p-x_q=\delta_x\) trigger compatibility). To obtain an MILP, linearize as follows:
    \begin{itemize}
      \item Replace each implication \(e_{uvr}=1 \Rightarrow x_v = x_u + \Delta_x(r)\) with big-M constraints:
      \[
      x_v - x_u - \Delta_x(r) \le M\,(1-e_{uvr}),\qquad
      x_v - x_u - \Delta_x(r) \ge -M\,(1-e_{uvr}),
      \]
      where \(M\) is a sufficiently large constant (e.g. \(M = m\)).
      \item For compatibility: for every possible displacement \((\delta_x,\delta_y)\) and for every ordered pair \((p,q)\), introduce binary indicator \(d_{pq}^{\delta}\) that equals 1 iff \((x_p-x_q,y_p-y_q)=(\delta_x,\delta_y)\). Enforce one-of constraints and link to the precomputed \(\mathrm{Comp}_{pq}^{\delta}\):
      \[
      d_{pq}^{\delta}=1 \Rightarrow \mathrm{Comp}_{pq}^{\delta}=1.
      \]
      Those implications again are linearizable via big-M.
    \end{itemize}
  \item The number of possible offsets \((\delta_x,\delta_y)\) is bounded: since all tiles must lie inside an \(m\times m\) canvas, \(|\delta_x|,|\delta_y|\le m-n\). Precomputing \(\mathrm{Comp}_{pq}^{\delta}\) for all feasible \(\delta\) reduces the on-line overlap checking cost.
  \item Enforcing a spanning tree (acyclic + connected) can be done with standard directed tree/arborescence constraints (e.g. parent variables plus subtour elimination or flow constraints) if you prefer connectivity instead of the simple `one parent per node` orientation.
\end{itemize}

\subsection*{Summary (informal)}
\begin{itemize}
  \item Decision variables place each tile at integer coordinates \((x_v,y_v)\) inside an \(m\times m\) canvas.
  \item A subset of relative placements \(e_{uvr}\) form a spanning tree; chosen edges force relative offsets between tiles and thus propagate absolute positions.
  \item Global compatibility requires that any two tiles that end up overlapping have equal symbols in all overlapped cells.
  \item Objective is to minimize \(m\).
\end{itemize}

\end{document}